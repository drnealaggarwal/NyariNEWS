\documentclass[a4paper,10pt, oneside]{letter}
\usepackage{block}
\usepackage{graphicx}
\usepackage{hyperref}
\signature{Nyari Residents Welfare Society\\ Nairobi, February 22nd, 2014}
\address{Nyari Residents Welfare Society \\ PO Box 407, Village Market, Nairobi 00621\\ Mobile: +254 738 543 865\\nyariresidents@gmail.com}

\begin{document}
\begin{letter} {}
\opening{Hi there Rahmat}

This is what I've come up with. Let me know if this is OK and edit the .tex file as needed and send back to me (don't remove the coding), if you want to make any changes. I've processed this file into the attached pdf file so you can see what it will look like as a simple pdf. I haven't had time to put it into our format; I'm run off my feet with work right now. Can we just send it out like this \ldots so much easier for me to do.\\

\newline
\hline

Dear Nyari Lake Resident

We, the Nyari Main committee are writing to you regarding latest developments on the lake and to inform you about some of the bye-laws of the estate with particular regard to those that govern the lake and the reason for their enactment. The lake is a highly sensitive enviroment subject to abuse through members' actions that you might not have considered. For this reason we have seperate guidelines and rules for the lake. They are as follows with explanations for their presence in greater list of Nyari bye-laws.

Please note: These are Nyari Residents Welfare Association (NRWS) bye-laws that are fully backed by the National Environment Welfare Authority (NEMA) and the Water Resource Management Authority (WRMA). As such we are ALL required to adhere to them. If any member fails to follow these rules and bye-laws the NRWS will have no recourse but to refer that member to the police, NEMA and/or WRMA. Please be advised that involvement of any of these authorities can cause the member concerned a great deal of trouble in extricating himself from the fines, court cases and monitoring that these authorities impose on recalcitrant persons.

\begin{enumerate}
\item No one is permitted on the lake after 6pm. The reason for this is two-fold. First: It compromises our security systems which require that no unknown or identifiable persons or vehicles be wandering around any part of the estate after sunset. Our guards have been instructed to arrest anyone found on the lake after 6pm. Please help us enforce this security constraint. Secondly, we have a large host of cattle egrets - the white birds - that roost on the north west end of the lake. These birds leave the lake in the early morning and come back to roost just after 6pm each evening. We will lose these birds if we have people on the lake frightening them away when they come to roost. Therefore please leave the lake before they arrive.

\item No fishing or boating in front of other plots except the one from which you originate. Maintain a distance of 25m (twenty five metres) minimum from all plot frontages. Certainly, your fishing lure should not be able to reach the reeds at the front of any persons plot. Please keep to the area close to your plot or the plot from which you entered the lake and certainly do not wander too close to any other persons plot. This rule exists in order to preserve our members' privacy. It is very disturbing to be sitting enjoying a quiet moment or some birds on the lake when a fishing vessel comes barging into your view, scaring away all the birds and then getting their lures stuck right in your field of vision. Please keep away from others plots.

\item Limit the number of visitors to your plot especially when you are not there to supervise them. All members living on the lake have agreed that the peace and quiet of the lake environment is paramount to them. For this reason please control the number of visitors and the noise that they generate. Note that according the city-wide bye-laws loud music is not permitted after 11pm. Disturbance of the peace through loud music will be reported to the police and if necessary to NEMA.

\item No person is permitted to fish who is not a member of Nyari or who has not explicitly been given permission to fish by a member of Nyari THAT LIVES ON THE LAKE. We constantly battle poachers -- we have had and continue to deal with people who come with a basket and fill it with fish to take to Gachie and neighbouring areas to sell on stalls there. Our lake is not able to take this kind of pressure. The fish control mosquito populations; overfish the lake and we will all suffer the nightmare of malaria. For the same reason please do not remove small fish or too many. Take a resonalble number of LARGE fish only. A resonable number is two (2) fish per fishing rod per day (this is a KWS standard for all lakes and rivers in Kenya).

\item Do not damage the reeds in any way. Reeds are important to prevent water loss by slowing down the wind speeds and also act as water purifiers by removing heavy metals and other toxins. The reed beds are also home to very many species of birds. If you get your lure stuck in the reeds please be very careful and quiet in approaching the reeds and removing the stuck lure.

\item Do not cut back or damage any plant life on or near the water for the same reasons as above. Do not cut any trees on the lake margins except with written permission of the NRWS committee who will only give such permission after careful consideration of the consequences of such tree cutting. Trees around the lake margin are critical for stabilizing the banks (especially the lake wall at the eastern end) for providing habitat for various bird species and for oxygenation that makes our lives that much more healthy. Do not cut down any trees.

\item No motor boats, boats with internal combustions engines or jet skis are permitted on the water. Electric motors and boats powered by oars (human power) or sails are permitted. Wind surfers, sailings dinghies and kite surfers are permitted. All internal combustion motors are prone to polluting the water with oil, fuel and other contaminants. They are also noisy and pollute the neighbourhood in this way too. For these reasons they are not permitted at any time.

\item Do not shine bright lights on the water they disturb your neighbours, scare away the bird life and prevent our guards seeing the water surface clearly at night.

\item Keep all septic tanks functioning properly. Overflowing effluent from septic tanks can lead to a Nairobi Dam scenario. We remind all of you that Nairobi Dam is not a solid piece of land and no longer a lake. We don't want this to happen to our lake. Not only will we lose this wonderful, peace-generating asset that we all share but also once it turns into solid ground you can be sure that a real estate developer will swoop in to put high rise aparments onto the newly-found land as soon as he can. Attempts to drain the lake for such nefarious use have already been thwarted in the past. Let's not revisit this nightmare scenario.

\item Do not use fertilizers closer than 25m to the water. This is a KWS directive and is clearly to prevent the kind of disaster we have seen at Nairobi Dam. Fertilizers cause alga to bloom in the water killing off the natural animals.

\item Do not use any chemicals near the water. Do not allow any paint or cement into the water. Again these chemicals kill off the natural organisms that are needed to keep the living ecosystem of the lake functioning normally.

\item Farm animals: Chickens, ducks, geese, goats -- these are considered farm animals and are not permitted withing city boundaries as per Nairobi City Council bye-laws. These animals are considered a health risk and are therefore ruled by the Council to be raised outside city limits. Nyari being well within city limits they are not permitted within the estate.

\item Do not pump lake water to water your gardens. The lake is at risk of drying up if you do this. Please use City Council water to water your gardens. As well as risking drying up of the lake the water from the lake is not suitable for drinking and contact with human skin can spread diseases that might be present in the lake. 

\item All persons use the lake entirely at their own risk. Nyari Residents Welfare Society is not responsible for any accident, illness or injury including death be that to a member or his visitor or to any other person whatsoever. In the past there have been three drownings on our lake and we have assisted in recovering the bodies but cannot be held responsible for providing life-guard-like facilities. Some people have swum in the lake and later been diagnosed with bilharzia. Handling of the water, even on fish bodies can cause a person to acquire this deadly disease. The NRWS is not responsible for this as it is impossible to treat the water. \\
\end{enumerate}

\closing{The Environment Sub-committee}
%\ps{}
%\encl{}
\end{letter}
\end{document}
